\documentclass[11pt,a4paper]{report}

% Paquetes necesarios

% Configuración de fuente Arial 11pt
% Para usar Arial real, compila con XeLaTeX en lugar de pdfLaTeX
\usepackage{fontspec}
\setmainfont{Arial}

\newfontfamily\bellmt{Bell MT}[
  ItalicFont = {Bell MT Italic},
]

\usepackage[spanish,es-tabla]{babel}
\usepackage[margin=2.54cm]{geometry} % Márgenes de 2.54cm
\usepackage{setspace} % Control de interlineado
\usepackage{graphicx} % Para incluir imágenes
\usepackage{fancyhdr} % Para encabezados y pies de página
\usepackage{titlesec} % Para formato de títulos
\usepackage{tocloft} % Para personalizar índices
\usepackage{indentfirst} % Sangría en primer párrafo
\usepackage{hyperref} % Enlaces en el PDF

% Configuración de hyperref para eliminar bordes rojos
\hypersetup{
    hidelinks
}
\usepackage{float} % Para posicionamiento de figuras
\usepackage{caption} % Para personalizar captions
\usepackage{subcaption} % Para subfiguras
\usepackage{array} % Para tablas
\usepackage{booktabs} % Para tablas profesionales
\usepackage{multirow} % Para celdas multirow en tablas
\usepackage{amsmath} % Para matemáticas
\usepackage{csquotes} % Para citas
\usepackage[style=apa,backend=biber]{biblatex} % Para bibliografía APA 7
\usepackage{appendix} % Para apéndices

% Configuración de bibliografía
\addbibresource{references.bib}

% Configuración de interlineado
\doublespacing % Interlineado doble

% Configuración de sangría
\setlength{\parindent}{1.27cm} % Sangría de primera línea

% Configuración de espaciado entre párrafos
\setlength{\parskip}{0pt} % Sin espacio extra entre párrafos

% Configuración de numeración de páginas
\pagestyle{fancy}
\fancyhf{}
\fancyhead[R]{\thepage} % Número de página en esquina superior derecha
\renewcommand{\headrulewidth}{0pt} % Sin línea en encabezado

% Forzar el estilo fancy en todas las páginas
\fancypagestyle{plain}{%
  \fancyhf{}
  \fancyhead[R]{\thepage}
  \renewcommand{\headrulewidth}{0pt}
}

% Configuración de formato de títulos según FORMAT.md
% Nivel 1: Bold, 11pt, Arial, centrado, sin numeración
\titleformat{\chapter}[display]
{\normalfont\bfseries\centering\fontsize{11}{13.2}\selectfont}{}{0pt}{}
\titlespacing*{\chapter}{0pt}{-30pt}{20pt}

% Nivel 2: Bold, 11pt, izquierda, con numeración
\titleformat{\section}
{\normalfont\bfseries\raggedright}{\thesection\quad}{0pt}{}
\titlespacing*{\section}{0pt}{0pt}{0pt}

% Nivel 3: Bold Italic, 11pt, izquierda, con numeración
\titleformat{\subsection}
{\normalfont\bfseries\itshape\raggedright}{\thesubsection\quad}{0pt}{}
\titlespacing*{\subsection}{0pt}{0pt}{0pt}

% Nivel 4: Bold, sangría 1.27cm, inline con punto
\titleformat{\subsubsection}[runin]
{\normalfont\bfseries}{}{0pt}{}[.]
\titlespacing*{\subsubsection}{\parindent}{0pt}{1em}

% Nivel 5: Bold Italic, sangría 1.27cm, inline con punto
\titleformat{\paragraph}[runin]
{\normalfont\bfseries\itshape}{}{0pt}{}[.]
\titlespacing*{\paragraph}{\parindent}{0pt}{1em}

% Configuración de captions para tablas y figuras (Arial 10pt exacto)
\DeclareCaptionFont{arial10}{\fontsize{10}{12}\selectfont}
\captionsetup{
    font={arial10,doublespacing},
    labelfont=bf,
    textfont=it,
    justification=centering,
    singlelinecheck=false
}

% Configuración de índices
\addto\captionsspanish{%
  \renewcommand{\contentsname}{Índice de contenido}%
  \renewcommand{\listtablename}{Índice de tablas}%
  \renewcommand{\listfigurename}{Índice de figuras}%
}

% Configuración del formato del título del índice
\makeatletter
\renewcommand\tableofcontents{%
    \chapter*{\centering\normalfont\bfseries\fontsize{11}{13.2}\selectfont\contentsname}%
    \@mkboth{\MakeUppercase\contentsname}{\MakeUppercase\contentsname}%
    \@starttoc{toc}%
}

% Configuración usando tocloft para los títulos de los índices
% Centrado perfecto, negrita, 11pt
\renewcommand{\cfttoctitlefont}{\hfill\normalfont\bfseries\fontsize{11}{13.2}\selectfont}
\renewcommand{\cftaftertoctitle}{\hfill\mbox{}}
\renewcommand{\cftlottitlefont}{\hfill\normalfont\bfseries\fontsize{11}{13.2}\selectfont}
\renewcommand{\cftafterlottitle}{\hfill\mbox{}}
\renewcommand{\cftloftitlefont}{\hfill\normalfont\bfseries\fontsize{11}{13.2}\selectfont}
\renewcommand{\cftafterloftitle}{\hfill\mbox{}}

% Eliminar el espacio vertical antes de los títulos de los índices
\setlength{\cftbeforetoctitleskip}{-30pt}
\setlength{\cftaftertoctitleskip}{20pt}
\setlength{\cftbeforelottitleskip}{-30pt}
\setlength{\cftafterlottitleskip}{20pt}
\setlength{\cftbeforeloftitleskip}{-30pt}
\setlength{\cftafterloftitleskip}{20pt}

\makeatother

% Configuración de puntitos (leaders) para todos los niveles
\renewcommand{\cftchapleader}{\cftdotfill{\cftdotsep}}
\setlength{\cftbeforechapskip}{0pt}

% Reducir la separación entre los puntos (valor por defecto es 4.5)
\renewcommand{\cftdotsep}{0}

% Comando para páginas sin numeración pero contadas
\newcommand{\unnumberedpage}{%
    \thispagestyle{empty}
    \addtocounter{page}{1}
}

% Comando para eliminar sangría (para secciones preliminares)
\newcommand{\noindentpages}{%
    \setlength{\parindent}{0pt}
}

% Comando para restaurar sangría (desde Introducción)
\newcommand{\restoreindent}{%
    \setlength{\parindent}{1.27cm}
}

% Documento principal
\begin{document}

% Carátula (sin numerar)
\thispagestyle{empty}
\begin{center}
    
    \includegraphics[width=3.5cm]{figures/logo_utpl.jpg} % Asegúrate de tener el logo
    
    \vspace{1cm}
    
    {\Large \textbf{UNIVERSIDAD TÉCNICA PARTICULAR DE LOJA}}\\
    {\large \textit{La Universidad Católica de Loja}}
    
    \vspace{1cm}
    
    {\large \textbf{FACULTAD DE XXXXXXX}}
    
    \vspace{1cm}
    
    {\large \textbf{MAESTRÍA EN XXXXXXXX}}
    
    \vspace{1cm}
    
    {\Large \textbf{Xxxxx xxxx xxxxxx xxxx}}
    
    \vspace{1cm}
    
    {\large Trabajo de titulación previo a la obtención del título de:}
    
    \vspace{1cm}
    
    {\large \textbf{XXXXXXX XXXXXXX}}
    
    \vspace{2cm}
    
    \begin{flushleft}
    \hspace{3cm}{\large \textbf{Autor:} Apellido1 Apellido2, Nombre1 Nombre2}\\
    \vspace{0.5cm}
    \hspace{3cm}{\large \textbf{Director:} Apellido1 Apellido2, Nombre1 Nombre2}
    \end{flushleft}
    
    \vfill
    
    {\large LOJA}\\
    {\large 2025}
    
    \end{center}
\clearpage

% Páginas preliminares con números romanos en mayúsculas (sin sangría)
\pagenumbering{Roman}
\setcounter{page}{1}
\noindentpages

% Aprobación del director (primera página numerada)
\addcontentsline{toc}{chapter}{Carátula}
\chapter*{Aprobación del director del Trabajo de Titulación}
\addcontentsline{toc}{chapter}{Aprobación del director del Trabajo de Titulación}

\vspace{1cm}

Loja, 15 de febrero de 2025

\vspace{1cm}

Doctor\\
Luis Rodrigo Barba Guamán\\
\textbf{Director de la maestría de Inteligencia Artificial Aplicada}\\

Loja.-\\

De mi consideración:

\vspace{0.5cm}

Me permito comunicar que, en calidad de director del presente Trabajo 
de Titulación nominado: Desarrollo de un prototipo móvil para la lectura
de medidores de agua potable usando técnicas de procesamiento de imágenes
realizado por Luis Antonio Pillaga Zhagñay ha sido orientado y revisado
durante su ejecución, así mismo ha sido verificado a través de la
herramienta de similitud académica institucional, y cuenta con un
porcentaje de coincidencia aceptable. En virtud de ello, y por considerar
que el mismo cumple con todos los parámetros establecidos por la
Universidad, doy mi aprobación a fin de continuar con el proceso
académico correspondiente.

\vspace{0.5cm}

Particular que comunico para los fines pertinentes.

\vspace{1cm}

Atentamente,

\vspace{3cm}

\noindent
Director: Guido Eduardo Riofrio Calderón, PhD.\\
C.I.: 1103214969\\
Correo electrónico: geriofrio@utpl.edu.ec
\clearpage

% Declaración de autoría
\chapter*{Declaración de autoría y cesión de derechos}
\addcontentsline{toc}{chapter}{Declaración de autoría y cesión de derechos}

Yo, Luis Antonio Pillaga Zhagñay, declaro y acepto en forma expresa lo siguiente:

Ser autor del Trabajo de Titulación denominado: Desarrollo de un prototipo móvil para la lectura de medidores de agua potable usando técnicas de procesamiento de imágenes, de la maestría en Inteligencia Artificial Aplicada, específicamente de los contenidos comprendidos en: Introducción, Capítulo 1: Marco Teórico, Capítulo 2: Metodología, Capítulo 3: Desarrollo y Resultados, Conclusiones y Recomendaciones, siendo Guido Eduardo Riofrio Calderón, director del presente trabajo; también declaro que la presente investigación no vulnera derechos de terceros ni utiliza fraudulentamente obras preexistentes. Además, ratifico que las ideas, criterios, opiniones, procedimientos y resultados vertidos en el presente trabajo investigativo, son de mi exclusiva responsabilidad. Eximo expresamente a la Universidad Técnica Particular de Loja y a sus representantes legales de posibles reclamos o acciones judiciales o administrativas, en relación a la propiedad intelectual de este trabajo.

Que la presente obra, producto de mis actividades académicas y de investigación, forma parte del patrimonio de la Universidad Técnica Particular de Loja, de conformidad con el artículo 20, literal j), de la Ley Orgánica de Educación Superior; y, artículo 91 del Estatuto Orgánico de la UTPL, que establece: ``Forman parte del patrimonio de la Universidad la propiedad intelectual de investigaciones, trabajos científicos o técnicos y tesis de grado que se realicen a través, o con el apoyo financiero, académico o institucional (operativo) de la Universidad'', en tal virtud, cedo a favor de la Universidad Técnica Particular de Loja la titularidad de los derechos patrimoniales que me corresponden en calidad de autor/a, de forma incondicional, completa, exclusiva y por todo el tiempo de su vigencia.

La Universidad Técnica Particular de Loja queda facultada para ingresar el presente trabajo al Sistema Nacional de Información de la Educación Superior del Ecuador para su difusión pública, en cumplimiento del artículo 144 de la Ley Orgánica de Educación Superior.

\vspace{2.5cm}

\noindent
Autor: Luis Antonio Pillaga Zhagñay\\
C.I.: 0301971495\\
Correo electrónico: lapillaga2@utpl.edu.ec
\clearpage

% Dedicatoria
\chapter*{Dedicatoria}
\addcontentsline{toc}{chapter}{Dedicatoria}

\vspace{2cm}

Xxxxxxxx xxxxxxxx xxxxxxxx xxxxxxxxxxxxxx xxxxxxx xxxxxxx xxxxxx xxxxxx xxxxxxxx xxxxxxxxxx xxxxxxxx xxxxxxxx xxxxxxxxx xxxxxxxxxxxxxxx xxxxxxx xxxxxxxxx.

Xxxxxxxx xxxxxxx xxxxxxxx xxxxxxx xxxxxxxx xxxxxxxxx xxxxxx xxxxx xxxxx xxxxxxxxxx xxxxxxx xxxxxxxxx xxxxxxx xxxxx xxxxxxxx.

Xxxxxxxx xxxxxxx xxxxxxxx xxxxxxx xxxxxxxx xxxxxxxxx xxxxxx xxxxx xxxxx xxxxxxxxxx xxxxxxx xxxxxxxxxxxxxx.
\clearpage

% Agradecimiento
% agradecimiento.tex
\chapter*{Agradecimiento}
\addcontentsline{toc}{chapter}{Agradecimiento}

TODO
\clearpage

% Índice de contenido
\tableofcontents
\clearpage

% Índice de tablas
\listoftables
\clearpage

% Índice de figuras
\listoffigures
\clearpage

% Cambio a números arábigos desde el resumen (sin sangría aún)
\pagenumbering{arabic}
\setcounter{page}{1}

% Resumen
% resumen.tex
\chapter*{Resumen}
\addcontentsline{toc}{chapter}{Resumen}

El resumen se presentará en un único párrafo con un máximo de \textbf{180 palabras}, sintetiza el aporte que brinda el trabajo realizado. \textbf{Obligatoriamente} debe contener las palabras clave \textbf{(máximo tres)}.

\vspace{0.5cm}

Ejemplo:

\vspace{0.5cm}

Xxxxxxxx xxxxxxxx xxxxxxxx xxxxxxxxxxxxxx xxxxxxx xxxxxxx xxxxxx xxxxxx xxxxxxxx xxxxxxxxxx xxxxxxxx xxxxxxxx xxxxxxxxx xxxxxxxxxxxxxxx xxxxxxx xxxxxxxxx xxxxxxx xxxxxxx xxxxxxxx xxxxxxx xxxxxxxx xxxxxxxxx xxxxxx xxxxx xxxxx xxxxxxxxxx xxxxxxx xxxxxxxxxx xxxxxxx xxxxxxxxxx xxxxxxxxx. Xxxxxxxx xxxxxxxx xxxxxxxx xxxxxxxxxxxxxx xxxxxxx xxxxxxx xxxxxx xxxxxx xxxxxxxxxxxxxxxxxxxxx xxxxxxxxxx xxxxxxxx xxxxxxxx xxxxxxxxx xxxxxxxxxxxxxxx xxxxxxx xxxxxxxxxxxxxxxx xxxxxxx xxxxxxxx xxxxxxx xxxxxxxx xxxxxxxxx xxxxxx xxxxx xxxxx xxxxxxxxxx xxxxxxx xxxxxxxxxx xxxxxxx xxxxxx xxxxx xxxxxxx xxxxxx xxxxxxxxxxxxxxxxxxx. Xxxxxxxx xxxxxxxxxxxx xxxxxxxxxxxxx xxxxxxxxxxxxxx xxxxxxx xxxxxxx xxxxxx xxxxxx xxxxxxxxxxxxxxx xxxxxxxxxxxxxxxx xxxxxxxx xxxxxxxx xxxxxxxxx xxxxxxxxxxxxxxx xxxxxxx xxxxxxxxxxxxxxxx xxxxxxx xxxxxxxx xxxxxxx xxxxxxxx xxxxxxxxx xxxxxx xxxxx xxxxx xxxxxxxxxx xxxxxxx xxxxxxxxxx xxxxxxxxxxxxxxxxxxxxxxx xxxxxxxxxxxxx xxxxxxxxxxxxxxxx. Xxxxxxxx xxxxxxxx xxxxxxxx xxxxxxxxxxxxxx xxxxxxx xxxxxxx xxxxxx xxxxxx xxxxxxxxxxxxxxxxxx xxxxxxxxxxxxxxxxxxxx xxxxxxxx xxxxxxxx xxxxxxxxx xxxxxxxxxxxxxxx xxxxxxx xxxxxxxxxxxxxxxx xxxxxxx xxxxxxxx xxxxxxx xxxxxxxx xxxxxxxxx xxxxxx xxxxx xxxxx xxxxxxxxxx xxxxxxx xxxxxxxxxx xx xxxxx xxxxxxx xxxxxx xxxxxxxxxxxxxxxx xxxxxxxxxxxxxx.

\vspace{0.5cm}

\textbf{\textit{Palabras clave:}} xxxxxxxxx, xxxxxxxx, xxxxxx.
\clearpage

% Abstract
\chapter*{Abstract}
\addcontentsline{toc}{chapter}{Abstract}

Abstract es el resumen traducido al idioma inglés en donde se incluyen las palabras claves. \textbf{Obligatoriamente} debe contener las palabras claves \textbf{(máximo tres)}.

\vspace{0.5cm}

Ejemplo:

\vspace{0.5cm}

Xxxxxxxx xxxxxxxx xxxxxxxx xxxxxxxxxxxxxx xxxxxxx xxxxxxx xxxxxx xxxxxx xxxxxxxx xxxxxxxxxx xxxxxxxx xxxxxxxx xxxxxxxxx xxxxxxxxxxxxxxx xxxxxxx xxxxxxxxx xxxxxxx xxxxxxx xxxxxxxx xxxxxxx xxxxxxxx xxxxxxxxx xxxxxx xxxxx xxxxx xxxxxxxxxx xxxxxxx xxxxxxxxxx xxxxxxx xxxxxxxxxx xxxxxxxxx. Xxxxxxxx xxxxxxxx xxxxxxxx xxxxxxxxxxxxxx xxxxxxx xxxxxxx xxxxxx xxxxxx xxxxxxxxxxxxxxxxxxxxx xxxxxxxxxx xxxxxxxx xxxxxxxx xxxxxxxxx xxxxxxxxxxxxxxx xxxxxxx xxxxxxxxxxxxxxxx xxxxxxx xxxxxxxx xxxxxxx xxxxxxxx xxxxxxxxx xxxxxx xxxxx xxxxx xxxxxxxxxx xxxxxxx xxxxxxxxxx xxxxxxx xxxxxx xxxxx xxxxxxx xxxxxx xxxxxxxxxxxxxxxxxxx. Xxxxxxxx xxxxxxxxxxxx xxxxxxxxxxxxx xxxxxxxxxxxxxx xxxxxxx xxxxxxx xxxxxx xxxxxx xxxxxxxxxxxxxxx xxxxxxxxxxxxxxxx xxxxxxxx xxxxxxxx xxxxxxxxx xxxxxxxxxxxxxxx xxxxxxx xxxxxxxxxxxxxxxx xxxxxxx xxxxxxxx xxxxxxx xxxxxxxx xxxxxxxxx xxxxxx xxxxx xxxxx xxxxxxxxxx xxxxxxx xxxxxxxxxx xxxxxxxxxxxxxxxxxxxxxxx xxxxxxxxxxxxx xxxxxxxxxxxxxxxx. Xxxxxxxx xxxxxxxx xxxxxxxx xxxxxxxxxxxxxx xxxxxxx xxxxxxx xxxxxx xxxxxx xxxxxxxxxxxxxxxxxx xxxxxxxxxxxxxxxxxxxx xxxxxxxx xxxxxxxx xxxxxxxxx xxxxxxxxxxxxxxx xxxxxxx xxxxxxxxxxxxxxxx xxxxxxx xxxxxxxx xxxxxxx xxxxxxxx xxxxxxxxx xxxxxx xxxxx xxxxx xxxxxxxxxx xxxxxxx xxxxxxxxxx xx xxxxx xxxxxxx xxxxxx xxxxxxxxxxxxxxxx xxxxxxxxxxxxxx.

\vspace{0.5cm}

\textbf{\textit{Keywords:}} xxxxxxxxx, xxxxxxxx, xxxxxx.
\clearpage

% Introducción (restaurar sangría desde aquí)
\restoreindent
% introduccion.tex
\chapter*{Introducción}
\addcontentsline{toc}{chapter}{Introducción}

Se sugiere presentar en máximo \textbf{dos páginas} y considerar los siguientes puntos:

\clearpage

% Capítulos
\chapter{Introducción}

\section{Contextualización del Problema: El Agua y Saneamiento en el Sector Rural}
\label{sec:context}

El acceso a agua potable y saneamiento constituye un derecho humano fundamental y es una piedra angular para el desarrollo sostenible, la salud pública y la prosperidad económica. La comunidad internacional ha ratificado su importancia a través de la Agenda 2030, donde el Objetivo de Desarrollo Sostenible 6 (ODS 6) insta a ``garantizar la disponibilidad y la gestión sostenible del agua y el saneamiento para todos'' \parencite{NacionesUnidas2015}. A pesar de los avances globales, persisten brechas significativas, especialmente en las zonas rurales de países en desarrollo, donde las comunidades a menudo dependen de modelos de gestión descentralizados y comunitarios para la prestación de estos servicios esenciales.

En el contexto de América Latina, y específicamente en Ecuador, las Juntas Administradoras de Agua Potable y Saneamiento (JAAPS) emergen como la estructura predominante para la gestión del agua en el sector rural. Estas organizaciones, de naturaleza comunitaria, son responsables de la administración, operación y mantenimiento de los sistemas de agua, desempeñando un rol crucial que el Estado no siempre logra cubrir \parencite{RoaGarcia2019}. Su labor no solo garantiza el suministro de agua a millones de personas, sino que también fomenta la cohesión social y la gobernanza local de los recursos naturales.

Sin embargo, la sostenibilidad de estas organizaciones enfrenta desafíos sistémicos. Estudios sobre la gestión comunitaria del agua en la región andina señalan que, si bien estos modelos son efectivos para expandir la cobertura, a menudo luchan con limitaciones técnicas, financieras y administrativas que comprometen su viabilidad a largo plazo \parencite{Toma2020}. La falta de herramientas tecnológicas adaptadas a su realidad operativa, caracterizada por recursos limitados y brechas de capacidad técnica, agrava estas dificultades y amenaza la calidad y continuidad del servicio, impactando directamente en el bienestar de las comunidades a las que sirven. Este trabajo de titulación se enmarca en la búsqueda de soluciones tecnológicas pertinentes y adaptadas para fortalecer a estas organizaciones vitales.

\section{Problemática}
\label{sec:problem_statement}

Si bien las JAAPS son un pilar en la gestión del agua rural, su modelo operativo tradicional se ve amenazado por un conjunto de ineficiencias críticas que comprometen su sostenibilidad financiera y la confianza de la comunidad. El núcleo de esta problemática reside en el proceso de toma de lecturas, una tarea manual que es fundamental para el ciclo de facturación y la gestión del recurso. Este proceso, que generalmente implica el desplazamiento de un operario para transcribir los valores de cada medidor en papel, es inherentemente lento y propenso a errores humanos que contribuyen directamente a las \textit{pérdidas comerciales} del sistema. Estas pérdidas aparentes, causadas por factores como errores en la lectura de medidores y en la manipulación de datos para la facturación, pueden generar una subvaloración significativa del consumo real, conduciendo a una facturación imprecisa y a la pérdida de ingresos vitales para la junta \parencite{ANC2025}.

Estas imprecisiones se ven agravadas por los desafíos logísticos. El personal de las JAAPS debe cubrir áreas geográficamente dispersas, a menudo con topografía compleja, lo que resulta en demoras operativas. Además, la dependencia de una conectividad a internet intermitente o inexistente en muchas de estas zonas rurales ha sido una barrera histórica para la adopción de soluciones digitales, que frecuentemente requieren una conexión constante para sincronizar datos con una plataforma centralizada \parencite{Fieldeke2021}.
Esta ``brecha digital'' obliga a las JAAPS a perpetuar flujos de trabajo basados en papel, con la consecuente duplicación de esfuerzos para la digitación de datos en sistemas contables, un paso que no solo consume tiempo valioso sino que también introduce una segunda fuente de posibles errores.

Las consecuencias de estas fallas operativas son sistémicas. Una facturación errónea erosiona la confianza de los usuarios en la administración de la junta, lo que puede derivar en una cultura de impago y afectar la recaudación. A su vez, la falta de datos de consumo fiables y oportunos impide a la directiva de la JAAPS realizar análisis técnicos esenciales, como el cálculo del balance hídrico o la detección temprana de fugas y consumos no autorizados, componentes clave del agua no contabilizada \parencite{ANC2025}. En última instancia, este círculo vicioso de ineficiencia administrativa y debilidad técnica limita la capacidad de las JAAPS para planificar inversiones, realizar mantenimientos preventivos y garantizar la sostenibilidad a largo plazo del servicio de agua potable para la comunidad.

\section{Justificación}
\label{sec:justification}

La presente investigación se justifica en tres dimensiones interconectadas que responden directamente a la problemática expuesta.

En primer lugar, desde una perspectiva tecnológica, el proyecto es oportuno y pertinente. Los avances recientes en el campo del aprendizaje profundo (deep learning), específicamente en arquitecturas de redes neuronales convolucionales (CNN) optimizadas para dispositivos de bajos recursos, han abierto la puerta a soluciones que antes eran inviables. El uso de frameworks como TensorFlow Lite permite encapsular modelos de visión por computadora potentes en un formato ligero, capaz de ejecutar la inferencia directamente en un teléfono móvil estándar sin depender de una conexión a internet \parencite{Carvalho2023}. Esta capacidad de procesamiento local (edge computing) es precisamente la innovación técnica que permite superar la barrera de la conectividad intermitente, el principal obstáculo para la digitalización de procesos en las zonas rurales atendidas por las JAAPS.

En segundo lugar, la justificación social y económica es directa y de alto impacto. Al automatizar el proceso de lectura de medidores, el prototipo propuesto tiene el potencial de reducir drásticamente los errores de facturación, incrementando la precisión y, por ende, la recuperación de ingresos para las juntas. Esta optimización de recursos no solo fortalece la sostenibilidad financiera de la organización, sino que también libera tiempo del personal, que puede ser reasignado a tareas críticas de mantenimiento y operación de la red. De esta manera, el proyecto se alinea con la Meta 6.B de la Agenda 2030 para el Desarrollo Sostenible, que promueve explícitamente el apoyo y fortalecimiento de la participación de las comunidades locales en la mejora de la gestión del agua y el saneamiento \parencite{NacionesUnidas2015}.

Finalmente, el proyecto presenta una clara justificación científica. Si bien existen soluciones genéricas de Reconocimiento Óptico de Caracteres (OCR), su rendimiento en condiciones no controladas (``in the wild'') es a menudo deficiente. La contribución de esta tesis radica en el desarrollo y la validación de un pipeline de visión artificial especializado para el dominio específico de los medidores de agua en contextos rurales, los cuales presentan una alta variabilidad de modelos, estados de deterioro, condiciones de iluminación y oclusiones parciales. La investigación aportará un dataset específico para esta tarea y analizará la eficacia de arquitecturas de modelos ligeros y técnicas de \textit{data augmentation} para lograr la alta precisión (\textgreater=98\%) requerida. Por lo tanto, el trabajo no solo resuelve un problema práctico, sino que también contribuye con conocimiento aplicable al campo del OCR en entornos industriales desafiantes y en hardware con recursos limitados.

\section{Objetivos}
\label{sec:objectives}

\subsection{Objetivo general}
\label{ssec:general_objective}

Desarrollar un prototipo móvil para la lectura automática de medidores de agua potable mediante técnicas de procesamiento de imágenes, que permita extraer indicadores de consumo con alta precisión y eficiencia.

\subsection{Objetivos específicos}
\label{ssec:specific_objectives}

\begin{enumerate}
\item Realizar una revisión sistemática de la literatura sobre técnicas de detección de objetos y reconocimiento óptico de caracteres (OCR) aplicadas a la lectura de medidores, con énfasis en modelos optimizados para dispositivos móviles.
\item Construir un dataset de imágenes de medidores de agua potable, capturadas en condiciones reales de las JAAPS, que sirva para el entrenamiento y la validación de los modelos de inteligencia artificial.
\item Desarrollar e implementar un modelo de aprendizaje profundo para la detección y lectura de los dígitos del medidor, optimizado para una alta precisión y una baja latencia de inferencia en un entorno móvil.
\item Diseñar y desarrollar una aplicación móvil para la plataforma Android que integre el modelo de IA y ofrezca un flujo de trabajo completo, desde la captura de la imagen hasta el registro de la lectura de forma local.
\item Validar el prototipo en condiciones de campo controladas, evaluando su precisión, eficiencia y usabilidad en comparación con el método de lectura manual tradicional utilizado por las JAAPS.
\end{enumerate}

\section{Alcance y Limitaciones}
\label{sec:scope_limitations}

A continuación, se definen las fronteras del presente trabajo de titulación, especificando tanto los entregables y funcionalidades que se desarrollarán, como los aspectos que quedan fuera del estudio.

\subsection{Alcance del Proyecto}
\label{ssec:scope}

El alcance de esta investigación cubre los siguientes puntos:
\begin{itemize}
\item El desarrollo de un prototipo funcional de una aplicación móvil para la plataforma Android, capaz de capturar imágenes de medidores de agua.
\item La creación de un dataset de imágenes propio, compuesto por fotografías de medidores de agua analógicos, recolectadas en condiciones de campo representativas de las Juntas Administradoras de Agua Potable y Saneamiento (JAAPS) rurales.
\item El entrenamiento y la optimización de un modelo de aprendizaje profundo (OCR), específicamente diseñado para la detección y lectura de los dígitos de los medidores de agua, y preparado para su ejecución local en el dispositivo móvil (offline).
\item La implementación de un flujo de trabajo en la aplicación que permite al usuario capturar la imagen, recibir la lectura procesada por el modelo de IA y registrarla de forma local en el dispositivo.
\item La validación técnica del prototipo en un entorno de campo controlado, donde se medirá y comparará su precisión y el tiempo de lectura frente al método manual tradicional.
\end{itemize}

\subsection{Limitaciones de la Investigación}
\label{ssec:limitations}

Para mantener el enfoque del proyecto, se establecen las siguientes limitaciones:
\begin{itemize}
\item El prototipo no es un sistema de gestión comercial completo. No incluirá funcionalidades avanzadas de facturación, gestión de abonados, o generación de reportes complejos, aunque su diseño permitirá la integración futura con dichos sistemas.
\item El modelo de OCR se especializará en los medidores de tipo analógico (de reloj o dial) identificados en el estudio. No se abordará la lectura de medidores digitales ni de modelos analógicos con características muy atípicas que no formen parte del dataset.
\item La aplicación móvil se desarrollará exclusivamente para la plataforma Android. No se crearán versiones para otros sistemas operativos como iOS.
\item La validación se centrará en la viabilidad y el rendimiento técnico (precisión, eficiencia) del prototipo. No se realizará un estudio a largo plazo sobre el impacto socioeconómico de la adopción de la herramienta en las JAAPS.
\item El despliegue y las pruebas de campo se realizarán con un número limitado y representativo de JAAPS, por lo que los resultados no pueden ser generalizados a todas las juntas del país sin estudios adicionales de escalabilidad.
\end{itemize}

\section{Estructura del Documento}
\label{sec:structure}

El presente trabajo de titulación se organiza en cinco capítulos, estructurados de la siguiente manera para presentar la investigación de forma lógica y coherente:

\begin{itemize}
\item \textbf{Capítulo 1: Introducción.} En el presente capítulo se ha contextualizado la problemática de la gestión del agua en las JAAPS rurales, se ha planteado el problema de la lectura manual de medidores y se ha justificado la necesidad de la investigación desde una perspectiva tecnológica, social y científica. Además, se han definido los objetivos, el alcance y las limitaciones del proyecto.

\item \textbf{Capítulo 2: Marco Teórico y Estado del Arte.} Este capítulo presentará una revisión sistemática de la literatura científica. Se abordarán los fundamentos de la visión por computadora, las arquitecturas de redes neuronales convolucionales para detección de objetos, y las técnicas de Reconocimiento Óptico de Caracteres (OCR), con un enfoque en su aplicación y optimización para dispositivos móviles.

\item \textbf{Capítulo 3: Metodología.} Se detallará el plan de trabajo y el diseño experimental de la investigación. Se describirá el proceso de recolección y anotación de imágenes para la creación del dataset, la selección de la arquitectura del modelo de IA, las métricas de evaluación, y el protocolo diseñado para la validación del prototipo en condiciones de campo.

\item \textbf{Capítulo 4: Desarrollo y Resultados.} En este capítulo se expondrá el proceso de desarrollo de la aplicación móvil y la implementación del modelo de IA. Posteriormente, se presentarán de manera objetiva los resultados obtenidos durante la fase de validación, incluyendo las métricas de precisión, eficiencia y una comparativa con el método manual.

\item \textbf{Capítulo 5: Discusión.} Se realizará un análisis crítico y una interpretación de los resultados presentados en el capítulo anterior, poniéndolos en diálogo con la literatura científica y los objetivos de la investigación. Se examinarán las implicaciones de los hallazgos y el valor de la solución propuesta.
\end{itemize}
\clearpage

\chapter{Marco Teórico}
\label{chap:marco_teorico}

\section{Gestión Comunitaria del Agua Potable y Saneamiento}
\label{sec:gestion_agua}
Esta sección contextualiza el ámbito de aplicación del proyecto, centrándose en el modelo de gestión del agua en las zonas rurales del Ecuador, el cual es operado por organizaciones comunitarias.

\subsection{Las Juntas Administradoras de Agua Potable y Saneamiento (JAAPS)}
\label{ssec:jaaps}
Aquí se profundizará en el rol de las JAAPS como entidades de gestión. Se analizará su marco legal, su estructura organizativa y los procesos operativos que rigen su funcionamiento, poniendo especial énfasis en los desafíos que enfrentan actualmente.

\subsubsection{Origen, marco legal y rol en el sector rural ecuatoriano}
\label{sssec:jaaps_origen}
Se investigará el marco normativo que regula a las JAAPS en Ecuador, su historia y su importancia estratégica para garantizar el acceso al agua potable en comunidades rurales, alineándose con los planes de desarrollo nacionales.

\subsubsection{Modelo de gestión: operación, administración y mantenimiento}
\label{sssec:jaaps_modelo}
Se describirá el ciclo de gestión típico de una JAAP, detallando los procesos de lectura de medidores, facturación, recaudación y mantenimiento de la infraestructura. El objetivo es identificar los cuellos de botella y las áreas de mejora.

\subsubsection{Problemáticas operativas: la lectura manual y sus implicaciones}
\label{sssec:jaaps_problemas}
Análisis específico de la problemática central: el proceso de lectura manual de medidores. Se cuantificarán y describirán las ineficiencias asociadas, como los errores de digitación, los altos tiempos operativos y los costos administrativos derivados.

\subsection{Hacia la Digitalización de los Servicios Básicos Rurales}
\label{ssec:digitalizacion_rural}
Se abordará la transformación digital como una herramienta para superar las brechas de eficiencia en el sector rural, vinculando la tecnología con los objetivos de desarrollo sostenible.

\subsubsection{Brecha digital en el contexto rural}
\label{sssec:brecha_digital}
Esta subsección explorará los desafíos de la implementación tecnológica en zonas rurales, considerando factores como la conectividad intermitente, la alfabetización digital y la disponibilidad de infraestructura.

\subsubsection{Objetivos de Desarrollo Sostenible (ODS 6) y su relación con la gestión eficiente del agua}
\label{sssec:ods6}
Se enmarcará el proyecto dentro de la Agenda 2030, específicamente en la Meta 6.B del ODS 6, que promueve el apoyo y fortalecimiento de la participación de las comunidades locales en la mejora de la gestión del agua y el saneamiento.

\section{Fundamentos de Visión por Computadora para el Análisis de Medidores}
\label{sec:vision_computadora}
Esta sección detalla las bases teóricas de la visión artificial que permitirán la correcta identificación y extracción de la lectura del medidor a partir de una imagen.

\subsection{Procesamiento Digital de Imágenes en Dispositivos Móviles}
\label{ssec:pdi_movil}
Se estudiarán las técnicas necesarias para preparar la imagen capturada por la cámara del móvil antes de ser procesada por el modelo de IA, asegurando la calidad y consistencia de la entrada.

\subsubsection{Adquisición y representación de la imagen: del sensor de la cámara al pre-procesamiento}
\label{sssec:adquisicion_imagen}
Se analizará cómo las imágenes son capturadas y representadas digitalmente, y las primeras etapas de normalización requeridas para el análisis computacional.

\subsubsection{Técnicas de pre-procesamiento para realce en condiciones de campo}
\label{sssec:preprocesamiento}
Investigación sobre algoritmos de mejora de imagen (ajuste de contraste, brillo, corrección de perspectiva, eliminación de ruido) cruciales para manejar las condiciones variables de campo (luz solar, sombras, reflejos).

\subsubsection{Segmentación de la región de interés (ROI): Localización del display del medidor}
\label{sssec:roi}
Se abordarán métodos para aislar automáticamente el área del contador numérico dentro de la imagen del medidor, separándola del resto del fondo para enfocar el análisis posterior.

\subsection{Modelos de Detección de Objetos}
\label{ssec:deteccion_objetos}
Se revisarán las arquitecturas de aprendizaje profundo especializadas en localizar objetos, que en este caso se aplicarán para encontrar con precisión la región del contador.

\subsubsection{Arquitecturas ligeras para inferencia en el borde (Edge AI)}
\label{sssec:edge_ai}
Estudio de modelos de redes neuronales (como MobileNet, EfficientDet) diseñados específicamente para ejecutarse de manera eficiente en dispositivos con recursos limitados, como los teléfonos móviles.

\subsubsection{Métricas de evaluación para modelos de detección}
\label{sssec:metricas_deteccion}
Se definirán y justificarán las métricas de rendimiento (como IoU - Intersection over Union y mAP - mean Average Precision) que se utilizarán para evaluar la precisión del modelo de detección de la ROI.

\section{Reconocimiento Óptico de Caracteres (OCR) Aplicado}
\label{sec:ocr}
Esta sección se centra en el núcleo de la solución: las técnicas de OCR que permitirán "leer" los dígitos del medidor una vez que la región de interés ha sido localizada.

\subsection{Evolución de las Técnicas de OCR}
\label{ssec:evolucion_ocr}
Se presentará una perspectiva histórica y conceptual del OCR, comparando los enfoques tradicionales con las soluciones modernas basadas en aprendizaje profundo para entender las ventajas de estas últimas.

\subsubsection{Métodos clásicos basados en análisis de contornos y plantillas}
\label{sssec:ocr_clasico}
Breve revisión de los algoritmos históricos de OCR para contextualizar el avance tecnológico en el área.

\subsubsection{OCR basado en Deep Learning: el estándar actual}
\label{sssec:ocr_dl}
Se explicará por qué las redes neuronales convolucionales y recurrentes han superado a los métodos clásicos, gracias a su capacidad para aprender características robustas directamente de los datos.

\subsection{Arquitecturas Neuronales para el Reconocimiento de Texto}
\label{ssec:arquitecturas_ocr}
Se investigarán las arquitecturas de redes neuronales específicas para la tarea de reconocer secuencias de caracteres, como los números en un medidor.

\subsubsection{Redes Neuronales Convolucionales (CNN) para la extracción de características de los dígitos}
\label{sssec:cnn_ocr}
Se describirá el papel de las CNN como extractores de características visuales de cada dígito individual en la imagen del contador.

\subsubsection{Modelos de secuencia (CRNN) para la lectura de los números del medidor}
\label{sssec:crnn}
Análisis de arquitecturas como CRNN (Convolutional Recurrent Neural Network) que combinan CNN con Redes Neuronales Recurrentes (RNN) para interpretar la secuencia de dígitos como un todo, mejorando la precisión contextual.

\subsection{Optimización de Modelos para Inferencia Local}
\label{ssec:optimizacion_modelos}
La ejecución en un móvil sin conexión a internet requiere modelos pequeños y rápidos. Esta sección explorará las técnicas para lograrlo.

\subsubsection{TensorFlow Lite: Cuantización, delegación y rendimiento en Android}
\label{sssec:tflite}
Investigación sobre el framework TensorFlow Lite, enfocándose en técnicas como la cuantización (reducción de la precisión de los pesos del modelo) y el uso de delegados (NNAPI, GPU) para acelerar la inferencia en el hardware de Android.

\subsubsection{Data Augmentation para robustecer el modelo ante variaciones en los medidores}
\label{sssec:data_augmentation}
Se estudiarán técnicas de aumento de datos (rotaciones, cambios de iluminación, zoom sintético) para crear un conjunto de entrenamiento más diverso y robusto a partir de un número limitado de imágenes reales.

\section{Desarrollo de Sistemas Móviles con Capacidades Offline}
\label{sec:desarrollo_movil}
Esta sección cubre los conceptos de desarrollo de software necesarios para construir el prototipo de aplicación móvil, con un enfoque en la funcionalidad offline.

\subsection{Arquitectura de Aplicaciones Android Modernas}
\label{ssec:arquitectura_android}
Se revisarán los patrones de diseño y las herramientas recomendadas para construir una aplicación Android robusta, escalable y mantenible.

\subsubsection{Componentes clave: Activities, ViewModels, CameraX para el control de la cámara}
\label{sssec:android_componentes}
Estudio de los componentes fundamentales del SDK de Android y las librerías de Jetpack (como CameraX y ViewModel) que facilitarán el desarrollo de la aplicación.

\subsection{Estrategias de Persistencia de Datos Offline-First}
\label{ssec:offline_first}
Dado que la conectividad en campo es poco fiable, la aplicación debe funcionar sin internet. Aquí se explorarán las estrategias para almacenar y gestionar los datos localmente.

\subsubsection{Bases de datos locales}
\label{sssec:db_local}
Análisis de las opciones de bases de datos embebidas en Android, principalmente SQLite y la capa de abstracción Room, para persistir las lecturas de los medidores y otros datos de la aplicación.

\subsubsection{Patrones de sincronización de datos para entornos con conectividad intermitente}
\label{sssec:sincronizacion}
Se investigarán estrategias y patrones de software para gestionar la sincronización de los datos locales con un servidor central cuando el dispositivo recupere la conexión a internet.

\section{Trabajos Relacionados y Estado del Arte}
\label{sec:estado_del_arte}
Esta sección final del marco teórico analizará la literatura científica y las soluciones existentes para la lectura automática de medidores, identificando las brechas y la contribución original de este proyecto.

\subsection{Sistemas de Lectura Automática de Medidores (AMR/AMI)}
\label{ssec:amr_ami}
Se revisarán los sistemas automáticos que dependen de hardware de comunicación especializado (como radiofrecuencia o PLC) para contextualizar dónde se ubica una solución basada en visión por computadora.

\subsection{Aplicaciones de Visión por Computadora para la lectura de medidores}
\label{ssec:cv_medidores}
Búsqueda y análisis de trabajos de investigación que hayan propuesto soluciones similares basadas en el procesamiento de imágenes para leer medidores de agua, gas o electricidad.

\subsection{Implementaciones de OCR en dispositivos móviles para casos de uso de campo}
\label{ssec:ocr_movil_campo}
Se estudiarán aplicaciones de OCR móvil en otros dominios (lectura de placas, escaneo de documentos) para extraer lecciones aprendidas sobre los desafíos de la implementación en condiciones de campo no controladas.

\subsection{Análisis comparativo de soluciones existentes}
\label{ssec:analisis_comparativo}
Se sintetizará la información recopilada, creando una tabla comparativa que evalúe las soluciones existentes en función de su precisión, velocidad, requisitos de hardware, dependencia de la nube y aplicabilidad al contexto de las JAAPS.
\clearpage


% Conclusiones
% conclusiones.tex
\chapter*{Conclusiones}
\addcontentsline{toc}{chapter}{Conclusiones}

Se redactan los puntos más sobresalientes, debilidades o fortalezas del proyecto o investigación, observados o descubiertos durante la ejecución del Trabajo de Integración Curricular, se recomienda redactar por cada conclusión, una recomendación.

Xxxxxxxxxxxxxxxxxxxxxxxxxxxxxxxxxxxxxxxxxxxxxxxxxxxxxxxxxxxxxxxxxxxxxxxxxxxxxxxxxxxxxxxxxxxxxxxxxxxxxxxxxxxxxxxxxxxxxxxxxxxxxxxxxxxxxxxxxxxxxxxxxxxxxxxxxxxxxxxxxxxxxxxxxxxxxxxxxxxxxxxxxxxxxxxxxxxxxxxxxxxxxxxxxxxxxxxxxxxxxxxxxxxxxxxxxxxxxxxxxxxxxxxxxxxxxxxxxxxxxxxxxxxxxxxxxxxxxxxxxxxxxxxxxxxxxxxxxxxxxxxxxxxxxxxxxxxxxxxxxxxxxxxxxxxxxxxxxxxxxxxxxxxxxxxxxxxxxxxxxxxxxxxxxxxxxxxxxxxxxxxxxxxxxxxxxxxxxxxxxxxxxxxxxxxxxxxxxxxxxxxxxxxxxxxxxxxxxxxxxxxxxxxxxxxxxxxxxxxxxxxxxx.

Xxxxxxxxxxxxxxxxxxxxxxxxxxxxxxxxxxxxxxxxxxxxxxxxxxxxxxxxxxxxxxxxxxxxxxxxxxxxxxxxxxxxxxxxxxxxxxxxxxxxxxxxxxxxxxxxxxxxxxxxxxxxxxxxxxxxxxxxxxxxxxxxxxxxxxxxxxxxxxxxxxxxxxxxxxxxxxxxxxxxxxxxxxxxxxxxxxxxxxxxxxxxxxxxxxxxxxxxxxxxxxxxxxxxxxxxxxxxxxxxxxxxxxxxxxxxxxxxxxxxxxxxxxxxxxxxxxxxxxxxxxxxxxxxxxxxxxxxxxxxxxxxxxxxxxxxxxxxxxxxxxxxxxxxxxxxxxxxxxxxxxxxxxxxxxxxxxxxxxxxxxxxxxxxxxxxxxxxxxxxxxxxxxxx.
\clearpage

% Recomendaciones
% recomendaciones.tex
\chapter*{Recomendaciones}
\addcontentsline{toc}{chapter}{Recomendaciones}

En esta parte debes sugerir temas para futuras investigaciones y puedan aportar a la academia.
\clearpage

% Referencias
\printbibliography[heading=bibintoc,title={Referencias}]
\clearpage


\end{document}