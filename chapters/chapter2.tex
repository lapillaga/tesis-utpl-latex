\chapter{Marco Teórico}
\label{chap:marco_teorico}

\section{Gestión Comunitaria del Agua Potable y Saneamiento}
\label{sec:gestion_agua}
Esta sección contextualiza el ámbito de aplicación del proyecto, centrándose en el modelo de gestión del agua en las zonas rurales del Ecuador, el cual es operado por organizaciones comunitarias.

\subsection{Las Juntas Administradoras de Agua Potable y Saneamiento (JAAPS)}
\label{ssec:jaaps}
Aquí se profundizará en el rol de las JAAPS como entidades de gestión. Se analizará su marco legal, su estructura organizativa y los procesos operativos que rigen su funcionamiento, poniendo especial énfasis en los desafíos que enfrentan actualmente.

\subsubsection{Origen, marco legal y rol en el sector rural ecuatoriano}
\label{sssec:jaaps_origen}
Se investigará el marco normativo que regula a las JAAPS en Ecuador, su historia y su importancia estratégica para garantizar el acceso al agua potable en comunidades rurales, alineándose con los planes de desarrollo nacionales.

\subsubsection{Modelo de gestión: operación, administración y mantenimiento}
\label{sssec:jaaps_modelo}
Se describirá el ciclo de gestión típico de una JAAP, detallando los procesos de lectura de medidores, facturación, recaudación y mantenimiento de la infraestructura. El objetivo es identificar los cuellos de botella y las áreas de mejora.

\subsubsection{Problemáticas operativas: la lectura manual y sus implicaciones}
\label{sssec:jaaps_problemas}
Análisis específico de la problemática central: el proceso de lectura manual de medidores. Se cuantificarán y describirán las ineficiencias asociadas, como los errores de digitación, los altos tiempos operativos y los costos administrativos derivados.

\subsection{Hacia la Digitalización de los Servicios Básicos Rurales}
\label{ssec:digitalizacion_rural}
Se abordará la transformación digital como una herramienta para superar las brechas de eficiencia en el sector rural, vinculando la tecnología con los objetivos de desarrollo sostenible.

\subsubsection{Brecha digital en el contexto rural}
\label{sssec:brecha_digital}
Esta subsección explorará los desafíos de la implementación tecnológica en zonas rurales, considerando factores como la conectividad intermitente, la alfabetización digital y la disponibilidad de infraestructura.

\subsubsection{Objetivos de Desarrollo Sostenible (ODS 6) y su relación con la gestión eficiente del agua}
\label{sssec:ods6}
Se enmarcará el proyecto dentro de la Agenda 2030, específicamente en la Meta 6.B del ODS 6, que promueve el apoyo y fortalecimiento de la participación de las comunidades locales en la mejora de la gestión del agua y el saneamiento.

\section{Fundamentos de Visión por Computadora para el Análisis de Medidores}
\label{sec:vision_computadora}
Esta sección detalla las bases teóricas de la visión artificial que permitirán la correcta identificación y extracción de la lectura del medidor a partir de una imagen.

\subsection{Procesamiento Digital de Imágenes en Dispositivos Móviles}
\label{ssec:pdi_movil}
Se estudiarán las técnicas necesarias para preparar la imagen capturada por la cámara del móvil antes de ser procesada por el modelo de IA, asegurando la calidad y consistencia de la entrada.

\subsubsection{Adquisición y representación de la imagen: del sensor de la cámara al pre-procesamiento}
\label{sssec:adquisicion_imagen}
Se analizará cómo las imágenes son capturadas y representadas digitalmente, y las primeras etapas de normalización requeridas para el análisis computacional.

\subsubsection{Técnicas de pre-procesamiento para realce en condiciones de campo}
\label{sssec:preprocesamiento}
Investigación sobre algoritmos de mejora de imagen (ajuste de contraste, brillo, corrección de perspectiva, eliminación de ruido) cruciales para manejar las condiciones variables de campo (luz solar, sombras, reflejos).

\subsubsection{Segmentación de la región de interés (ROI): Localización del display del medidor}
\label{sssec:roi}
Se abordarán métodos para aislar automáticamente el área del contador numérico dentro de la imagen del medidor, separándola del resto del fondo para enfocar el análisis posterior.

\subsection{Modelos de Detección de Objetos}
\label{ssec:deteccion_objetos}
Se revisarán las arquitecturas de aprendizaje profundo especializadas en localizar objetos, que en este caso se aplicarán para encontrar con precisión la región del contador.

\subsubsection{Arquitecturas ligeras para inferencia en el borde (Edge AI)}
\label{sssec:edge_ai}
Estudio de modelos de redes neuronales (como MobileNet, EfficientDet) diseñados específicamente para ejecutarse de manera eficiente en dispositivos con recursos limitados, como los teléfonos móviles.

\subsubsection{Métricas de evaluación para modelos de detección}
\label{sssec:metricas_deteccion}
Se definirán y justificarán las métricas de rendimiento (como IoU - Intersection over Union y mAP - mean Average Precision) que se utilizarán para evaluar la precisión del modelo de detección de la ROI.

\section{Reconocimiento Óptico de Caracteres (OCR) Aplicado}
\label{sec:ocr}
Esta sección se centra en el núcleo de la solución: las técnicas de OCR que permitirán "leer" los dígitos del medidor una vez que la región de interés ha sido localizada.

\subsection{Evolución de las Técnicas de OCR}
\label{ssec:evolucion_ocr}
Se presentará una perspectiva histórica y conceptual del OCR, comparando los enfoques tradicionales con las soluciones modernas basadas en aprendizaje profundo para entender las ventajas de estas últimas.

\subsubsection{Métodos clásicos basados en análisis de contornos y plantillas}
\label{sssec:ocr_clasico}
Breve revisión de los algoritmos históricos de OCR para contextualizar el avance tecnológico en el área.

\subsubsection{OCR basado en Deep Learning: el estándar actual}
\label{sssec:ocr_dl}
Se explicará por qué las redes neuronales convolucionales y recurrentes han superado a los métodos clásicos, gracias a su capacidad para aprender características robustas directamente de los datos.

\subsection{Arquitecturas Neuronales para el Reconocimiento de Texto}
\label{ssec:arquitecturas_ocr}
Se investigarán las arquitecturas de redes neuronales específicas para la tarea de reconocer secuencias de caracteres, como los números en un medidor.

\subsubsection{Redes Neuronales Convolucionales (CNN) para la extracción de características de los dígitos}
\label{sssec:cnn_ocr}
Se describirá el papel de las CNN como extractores de características visuales de cada dígito individual en la imagen del contador.

\subsubsection{Modelos de secuencia (CRNN) para la lectura de los números del medidor}
\label{sssec:crnn}
Análisis de arquitecturas como CRNN (Convolutional Recurrent Neural Network) que combinan CNN con Redes Neuronales Recurrentes (RNN) para interpretar la secuencia de dígitos como un todo, mejorando la precisión contextual.

\subsection{Optimización de Modelos para Inferencia Local}
\label{ssec:optimizacion_modelos}
La ejecución en un móvil sin conexión a internet requiere modelos pequeños y rápidos. Esta sección explorará las técnicas para lograrlo.

\subsubsection{TensorFlow Lite: Cuantización, delegación y rendimiento en Android}
\label{sssec:tflite}
Investigación sobre el framework TensorFlow Lite, enfocándose en técnicas como la cuantización (reducción de la precisión de los pesos del modelo) y el uso de delegados (NNAPI, GPU) para acelerar la inferencia en el hardware de Android.

\subsubsection{Data Augmentation para robustecer el modelo ante variaciones en los medidores}
\label{sssec:data_augmentation}
Se estudiarán técnicas de aumento de datos (rotaciones, cambios de iluminación, zoom sintético) para crear un conjunto de entrenamiento más diverso y robusto a partir de un número limitado de imágenes reales.

\section{Desarrollo de Sistemas Móviles con Capacidades Offline}
\label{sec:desarrollo_movil}
Esta sección cubre los conceptos de desarrollo de software necesarios para construir el prototipo de aplicación móvil, con un enfoque en la funcionalidad offline.

\subsection{Arquitectura de Aplicaciones Android Modernas}
\label{ssec:arquitectura_android}
Se revisarán los patrones de diseño y las herramientas recomendadas para construir una aplicación Android robusta, escalable y mantenible.

\subsubsection{Componentes clave: Activities, ViewModels, CameraX para el control de la cámara}
\label{sssec:android_componentes}
Estudio de los componentes fundamentales del SDK de Android y las librerías de Jetpack (como CameraX y ViewModel) que facilitarán el desarrollo de la aplicación.

\subsection{Estrategias de Persistencia de Datos Offline-First}
\label{ssec:offline_first}
Dado que la conectividad en campo es poco fiable, la aplicación debe funcionar sin internet. Aquí se explorarán las estrategias para almacenar y gestionar los datos localmente.

\subsubsection{Bases de datos locales}
\label{sssec:db_local}
Análisis de las opciones de bases de datos embebidas en Android, principalmente SQLite y la capa de abstracción Room, para persistir las lecturas de los medidores y otros datos de la aplicación.

\subsubsection{Patrones de sincronización de datos para entornos con conectividad intermitente}
\label{sssec:sincronizacion}
Se investigarán estrategias y patrones de software para gestionar la sincronización de los datos locales con un servidor central cuando el dispositivo recupere la conexión a internet.

\section{Trabajos Relacionados y Estado del Arte}
\label{sec:estado_del_arte}
Esta sección final del marco teórico analizará la literatura científica y las soluciones existentes para la lectura automática de medidores, identificando las brechas y la contribución original de este proyecto.

\subsection{Sistemas de Lectura Automática de Medidores (AMR/AMI)}
\label{ssec:amr_ami}
Se revisarán los sistemas automáticos que dependen de hardware de comunicación especializado (como radiofrecuencia o PLC) para contextualizar dónde se ubica una solución basada en visión por computadora.

\subsection{Aplicaciones de Visión por Computadora para la lectura de medidores}
\label{ssec:cv_medidores}
Búsqueda y análisis de trabajos de investigación que hayan propuesto soluciones similares basadas en el procesamiento de imágenes para leer medidores de agua, gas o electricidad.

\subsection{Implementaciones de OCR en dispositivos móviles para casos de uso de campo}
\label{ssec:ocr_movil_campo}
Se estudiarán aplicaciones de OCR móvil en otros dominios (lectura de placas, escaneo de documentos) para extraer lecciones aprendidas sobre los desafíos de la implementación en condiciones de campo no controladas.

\subsection{Análisis comparativo de soluciones existentes}
\label{ssec:analisis_comparativo}
Se sintetizará la información recopilada, creando una tabla comparativa que evalúe las soluciones existentes en función de su precisión, velocidad, requisitos de hardware, dependencia de la nube y aplicabilidad al contexto de las JAAPS.